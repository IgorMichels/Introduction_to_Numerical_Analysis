\documentclass{article}
\usepackage[utf8]{inputenc}
\usepackage[portuguese]{babel}
\usepackage{hyperref}
\usepackage{geometry}
\usepackage{amsmath}
\usepackage{amsfonts}
\usepackage{amssymb}
\usepackage{dsfont}
\usepackage{indentfirst}
\usepackage{listings}
\usepackage{xcolor}
\usepackage{graphicx}
\usepackage{float}
\usepackage{multicol}
\usepackage{halloweenmath}

\definecolor{codegreen}{rgb}{0,0.6,0}
\definecolor{codegray}{rgb}{0.5,0.5,0.5}
\definecolor{codepurple}{rgb}{0.58,0,0.82}
\definecolor{backcolour}{rgb}{0.95,0.95,0.92}

\lstdefinestyle{mystyle}{
    backgroundcolor=\color{backcolour},   
    commentstyle=\color{codegreen},
    keywordstyle=\color{magenta},
    numberstyle=\tiny\color{codegray},
    stringstyle=\color{codepurple},
    basicstyle=\ttfamily\footnotesize,
    breakatwhitespace=false,         
    breaklines=true,                 
    captionpos=b,                    
    keepspaces=true,                 
    numbers=left,                    
    numbersep=5pt,                  
    showspaces=false,                
    showstringspaces=false,
    showtabs=false,                  
    tabsize=2
}

\lstset{style=mystyle}
\geometry{top = 3cm, bottom = 2cm, left = 3cm, right = 2cm}

\title{Introdução à Análise Numérica \\ 3ª Lista}
\author{Igor Patrício Michels}
\date{06/10/2021}

\begin{document}

\maketitle

Primeiramente, note que, $x^*$ é a raiz de $f$ se, e somente se, $f(x^*) = 0$, ou seja, $x^* = \ln{\left(15 - \ln{\left(x^*\right)}\right)}$, ou seja, $x^*$ é ponto fixo da aplicação $g(x) = \ln{\left(15 - \ln{\left(x\right)}\right)}$.

Pelo Teorema de Lagrange, sabemos que, para todo $x, y\in [1, 3]$, vale que
\[|g(x) - g(y)| = |g'(c)| |x - y|, \text{ para algum } c\in (x, y).\]

Notemos que
\[g'(x) = -\dfrac{1}{x\left(15 - \ln{\left(x\right)})\right)}\]

Como sabemos que $\ln{\left(x\right)}$ é uma função monótona, temos que $15 - \ln{\left(x\right)}$ também é e, consequentemente, $g(x)$ será monótona também. Dessa forma, podemos afirmar que $g'(x) \in \left[-\frac{1}{15}, \frac{1}{\ln{(27)} - 45}\right]$, ou seja, $|g'(x)| < 1$ para todo $x\in [1, 3]$, o que mostra que a função é contrativa. Para mostrar que $g\left([1, 3]\right)\subseteq [1, 3]$, podemos usar o fato de $g(x)$ ser monótona e avaliar apenas os extremos da função: $g(1) = \ln{(15)}\approx 2.708$ e $g(3) = \ln{\left(15 - \ln{(3)}\right)}\approx 2.632$, ou seja, podemos afirmar que $g\left([1, 3]\right)\subseteq [1, 3]$.

Com isso, podemos afirmar que, dado qualquer $x_0\in [1, 3]$, a sequência dada por $x_{n + 1} = g(x_n)$ converge para $x^*$ tal que $x^* = g(x^*)$, isso é, para o ponto fixo de $g$, o qual é a raiz de $f$.

Para encontrar $x^*$ por meio dessa recorrência, usando $x_0 = 2$ e precisão de $5$ algarismos significativos, são necessárias $4$ iterações e obtemos $x^* = 2.6411119626298705$. Já utilizando o Método de Newton-Raphson, para $x_0 = 2$ e precisão de $8$ algarismos significativos, temos $x^* = 2.641112347187374$ após $3$ iterações.

\end{document}
